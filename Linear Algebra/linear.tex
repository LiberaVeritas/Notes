\documentclass[../main.tex]{subfiles}

\title{Linear Algebra}
\author{}
\date{}

\begin{document}

\maketitle
\tableofcontents

\newpage

\section{Vectors}

Vectors are a way of describing things along different dimensions or aspects.
Each coordinate of the vector describes the quantity along each dimension.
The standard basis vectors \( e_i \)
are vectors whose quantities are just 1 along the dimension \( i \)
and zero for the others.
Thus,
any vector \( v \) is just the linear combination of its coordinates
with the corresponding basis vectors
\( v_1 e_1 + \dots + v_n e_n \).

Geometrically, we can think of a vector \( \vb{v} \) as a position in \( \R^n \),
whose position along each axis is specified by the coordinates of \( \vb{v} \).

Another way to think about a vector is as a displacement.
For example,
we can think of \( \va{v} \) as the change from the position \( \vb{0} \) to \( \vb{v} \),
or \( \va{v} = \vb{v} - \vb{0} \).

\noindent
\begin{tikzpicture}[x=0.75pt,y=0.75pt,yscale=-1,xscale=1]
%uncomment if require: \path (0,300); %set diagram left start at 0, and has height of 300

%Shape: Circle [id:dp6722596476060768] 
\draw   (103,86.25) .. controls (103,78.93) and (108.93,73) .. (116.25,73) .. controls (123.57,73) and (129.5,78.93) .. (129.5,86.25) .. controls (129.5,93.57) and (123.57,99.5) .. (116.25,99.5) .. controls (108.93,99.5) and (103,93.57) .. (103,86.25) -- cycle ;
%Shape: Circle [id:dp74444730195737] 
\draw   (140,86.25) .. controls (140,78.93) and (145.93,73) .. (153.25,73) .. controls (160.57,73) and (166.5,78.93) .. (166.5,86.25) .. controls (166.5,93.57) and (160.57,99.5) .. (153.25,99.5) .. controls (145.93,99.5) and (140,93.57) .. (140,86.25) -- cycle ;
%Shape: Circle [id:dp9832620880412448] 
\draw   (230,86.25) .. controls (230,78.93) and (235.93,73) .. (243.25,73) .. controls (250.57,73) and (256.5,78.93) .. (256.5,86.25) .. controls (256.5,93.57) and (250.57,99.5) .. (243.25,99.5) .. controls (235.93,99.5) and (230,93.57) .. (230,86.25) -- cycle ;
%Shape: Brace [id:dp5022040197364186] 
\draw   (236.5,57) .. controls (236.5,52.33) and (234.17,50) .. (229.5,50) -- (188,50) .. controls (181.33,50) and (178,47.67) .. (178,43) .. controls (178,47.67) and (174.67,50) .. (168,50)(171,50) -- (126.5,50) .. controls (121.83,50) and (119.5,52.33) .. (119.5,57) ;
%Shape: Square [id:dp23486906391618334] 
\draw   (100,171) -- (126.5,171) -- (126.5,197.5) -- (100,197.5) -- cycle ;
%Shape: Square [id:dp8514361205022334] 
\draw   (134,171) -- (160.5,171) -- (160.5,197.5) -- (134,197.5) -- cycle ;
%Shape: Square [id:dp5165693986546934] 
\draw   (226,172) -- (252.5,172) -- (252.5,198.5) -- (226,198.5) -- cycle ;
%Shape: Brace [id:dp2319346457546657] 
\draw   (110.5,221) .. controls (110.5,225.67) and (112.83,228) .. (117.5,228) -- (169,228) .. controls (175.67,228) and (179,230.33) .. (179,235) .. controls (179,230.33) and (182.33,228) .. (189,228)(186,228) -- (240.5,228) .. controls (245.17,228) and (247.5,225.67) .. (247.5,221) ;
%Straight Lines [id:da9665534418589774] 
\draw    (113.25,170) -- (113.25,131) ;
\draw [shift={(113.25,129)}, rotate = 450] [color={rgb, 255:red, 0; green, 0; blue, 0 }  ][line width=0.75]    (10.93,-3.29) .. controls (6.95,-1.4) and (3.31,-0.3) .. (0,0) .. controls (3.31,0.3) and (6.95,1.4) .. (10.93,3.29)   ;
%Straight Lines [id:da2232187544111287] 
\draw    (148.25,171) -- (148.25,132) ;
\draw [shift={(148.25,130)}, rotate = 450] [color={rgb, 255:red, 0; green, 0; blue, 0 }  ][line width=0.75]    (10.93,-3.29) .. controls (6.95,-1.4) and (3.31,-0.3) .. (0,0) .. controls (3.31,0.3) and (6.95,1.4) .. (10.93,3.29)   ;
%Straight Lines [id:da8929455340252761] 
\draw    (240.25,169) -- (240.25,130) ;
\draw [shift={(240.25,128)}, rotate = 450] [color={rgb, 255:red, 0; green, 0; blue, 0 }  ][line width=0.75]    (10.93,-3.29) .. controls (6.95,-1.4) and (3.31,-0.3) .. (0,0) .. controls (3.31,0.3) and (6.95,1.4) .. (10.93,3.29)   ;

% Text Node
\draw (185,86.4) node [anchor=north west][inner sep=0.75pt]    {$\dotsc $};
% Text Node
\draw (170,14.4) node [anchor=north west][inner sep=0.75pt]    {$n$};
% Text Node
\draw (180,187.4) node [anchor=north west][inner sep=0.75pt]    {$\dotsc $};
% Text Node
\draw (175,248.4) node [anchor=north west][inner sep=0.75pt]    {$r$};


\end{tikzpicture}


\subsection{Addition and subtraction}

Adding a vector to another is simply just adding the quantities along each dimension.
Subtraction is taking the difference.

Geometrically, we can add two vectors \( \vb{u} \) and \( \vb{v} \)
by considering where we end up if we start at position \( \vb{u} \)
then apply the change \( \va{v} \).
Thus, \( \vb{u} + \va{v} = \vb{z} \).

\noindent
\tikzset{every picture/.style={line width=0.75pt}} %set default line width to 0.75pt        

\begin{tikzpicture}[x=0.75pt,y=0.75pt,yscale=-1,xscale=1]
%uncomment if require: \path (0,300); %set diagram left start at 0, and has height of 300

%Shape: Axis 2D [id:dp7522167816409159] 
\draw  (50,242) -- (419.5,242)(97.5,8) -- (97.5,267) (412.5,237) -- (419.5,242) -- (412.5,247) (92.5,15) -- (97.5,8) -- (102.5,15) (146.5,237) -- (146.5,247)(195.5,237) -- (195.5,247)(244.5,237) -- (244.5,247)(293.5,237) -- (293.5,247)(342.5,237) -- (342.5,247)(391.5,237) -- (391.5,247)(92.5,193) -- (102.5,193)(92.5,144) -- (102.5,144)(92.5,95) -- (102.5,95)(92.5,46) -- (102.5,46) ;
\draw   ;
%Shape: Circle [id:dp1800333897763694] 
\draw   (297,120.75) .. controls (297,117.57) and (299.57,115) .. (302.75,115) .. controls (305.93,115) and (308.5,117.57) .. (308.5,120.75) .. controls (308.5,123.93) and (305.93,126.5) .. (302.75,126.5) .. controls (299.57,126.5) and (297,123.93) .. (297,120.75) -- cycle ;
%Curve Lines [id:da6744184215507455] 
\draw    (98,171) .. controls (107.79,163.66) and (111.03,107.89) .. (139.27,154.95) .. controls (167.5,202) and (206.02,160.39) .. (242.5,187) .. controls (278.98,213.61) and (267.5,117) .. (297,120.75) ;
%Straight Lines [id:da042236632067626734] 
\draw [color={rgb, 255:red, 255; green, 0; blue, 0 }  ,draw opacity=1 ]   (302,103) -- (302,62) ;
\draw [shift={(302,60)}, rotate = 450] [color={rgb, 255:red, 255; green, 0; blue, 0 }  ,draw opacity=1 ][line width=0.75]    (10.93,-3.29) .. controls (6.95,-1.4) and (3.31,-0.3) .. (0,0) .. controls (3.31,0.3) and (6.95,1.4) .. (10.93,3.29)   ;
%Straight Lines [id:da8879089279998805] 
\draw [color={rgb, 255:red, 255; green, 0; blue, 0 }  ,draw opacity=1 ]   (301.75,136.5) -- (301.75,175) ;
\draw [shift={(301.75,177)}, rotate = 270] [color={rgb, 255:red, 255; green, 0; blue, 0 }  ,draw opacity=1 ][line width=0.75]    (10.93,-3.29) .. controls (6.95,-1.4) and (3.31,-0.3) .. (0,0) .. controls (3.31,0.3) and (6.95,1.4) .. (10.93,3.29)   ;
%Straight Lines [id:da984234265425485] 
\draw  [dash pattern={on 0.84pt off 2.51pt}]  (302.75,126.5) -- (302.75,245) ;
%Straight Lines [id:da851126430686476] 
\draw [color={rgb, 255:red, 0; green, 42; blue, 255 }  ,draw opacity=1 ]   (267,260) -- (298.5,260) ;
\draw [shift={(300.5,260)}, rotate = 180] [color={rgb, 255:red, 0; green, 42; blue, 255 }  ,draw opacity=1 ][line width=0.75]    (10.93,-3.29) .. controls (6.95,-1.4) and (3.31,-0.3) .. (0,0) .. controls (3.31,0.3) and (6.95,1.4) .. (10.93,3.29)   ;


\end{tikzpicture}

Since \( \vb{u} \) is also a change from \( \vb{0} \),
we can think of \( \va u + \va v \) as the composition of two changes as well.

We have already seen vector subtraction.
We see that \( \va{z} = \vb{u} - \vb{v} \)
is just the change in position from \( \vb{v} \) to \( \vb{u} \).

\noindent
\tikzset{every picture/.style={line width=0.75pt}} %set default line width to 0.75pt        

\noindent
\begin{tikzpicture}[x=0.75pt,y=0.75pt,yscale=-1,xscale=1]
%uncomment if require: \path (0,300); %set diagram left start at 0, and has height of 300

%Shape: Axis 2D [id:dp049849838656989] 
\draw  (70,180) -- (535.5,180)(183.5,60) -- (183.5,282) (528.5,175) -- (535.5,180) -- (528.5,185) (178.5,67) -- (183.5,60) -- (188.5,67) (216.5,175) -- (216.5,185)(249.5,175) -- (249.5,185)(282.5,175) -- (282.5,185)(315.5,175) -- (315.5,185)(348.5,175) -- (348.5,185)(381.5,175) -- (381.5,185)(414.5,175) -- (414.5,185)(447.5,175) -- (447.5,185)(480.5,175) -- (480.5,185)(513.5,175) -- (513.5,185)(150.5,175) -- (150.5,185)(117.5,175) -- (117.5,185)(84.5,175) -- (84.5,185)(178.5,147) -- (188.5,147)(178.5,114) -- (188.5,114)(178.5,81) -- (188.5,81)(178.5,213) -- (188.5,213)(178.5,246) -- (188.5,246) ;
\draw   ;
%Straight Lines [id:da9974234043672462] 
\draw    (183.5,180) -- (276.05,92.38) ;
\draw [shift={(277.5,91)}, rotate = 496.57] [color={rgb, 255:red, 0; green, 0; blue, 0 }  ][line width=0.75]    (10.93,-3.29) .. controls (6.95,-1.4) and (3.31,-0.3) .. (0,0) .. controls (3.31,0.3) and (6.95,1.4) .. (10.93,3.29)   ;
%Straight Lines [id:da20657379122009134] 
\draw    (183.5,180) -- (273.67,219.2) ;
\draw [shift={(275.5,220)}, rotate = 203.5] [color={rgb, 255:red, 0; green, 0; blue, 0 }  ][line width=0.75]    (10.93,-3.29) .. controls (6.95,-1.4) and (3.31,-0.3) .. (0,0) .. controls (3.31,0.3) and (6.95,1.4) .. (10.93,3.29)   ;
%Shape: Arc [id:dp5803037905068104] 
\draw  [draw opacity=0] (200.9,165.06) .. controls (203.44,166.14) and (205.18,168.69) .. (205.1,171.61) .. controls (205.01,174.63) and (202.98,177.14) .. (200.25,177.99) -- (198.21,171.4) -- cycle ; \draw   (200.9,165.06) .. controls (203.44,166.14) and (205.18,168.69) .. (205.1,171.61) .. controls (205.01,174.63) and (202.98,177.14) .. (200.25,177.99) ;
%Shape: Arc [id:dp7047653051043884] 
\draw  [draw opacity=0] (207.18,181.49) .. controls (209,182.26) and (210.25,184.09) .. (210.19,186.18) .. controls (210.12,188.34) and (208.68,190.13) .. (206.72,190.74) -- (205.26,186.03) -- cycle ; \draw   (207.18,181.49) .. controls (209,182.26) and (210.25,184.09) .. (210.19,186.18) .. controls (210.12,188.34) and (208.68,190.13) .. (206.72,190.74) ;
%Straight Lines [id:da4324446373354387] 
\draw [color={rgb, 255:red, 255; green, 0; blue, 0 }  ,draw opacity=1 ]   (183.5,180) -- (513.56,97.49) ;
\draw [shift={(515.5,97)}, rotate = 525.96] [color={rgb, 255:red, 255; green, 0; blue, 0 }  ,draw opacity=1 ][line width=0.75]    (10.93,-3.29) .. controls (6.95,-1.4) and (3.31,-0.3) .. (0,0) .. controls (3.31,0.3) and (6.95,1.4) .. (10.93,3.29)   ;
%Shape: Arc [id:dp5325215380673752] 
\draw  [draw opacity=0] (235.18,168.49) .. controls (237,169.26) and (238.25,171.09) .. (238.19,173.18) .. controls (238.12,175.34) and (236.68,177.13) .. (234.72,177.74) -- (233.26,173.03) -- cycle ; \draw  [color={rgb, 255:red, 255; green, 0; blue, 0 }  ,draw opacity=1 ] (235.18,168.49) .. controls (237,169.26) and (238.25,171.09) .. (238.19,173.18) .. controls (238.12,175.34) and (236.68,177.13) .. (234.72,177.74) ;

% Text Node
\draw (250,72.4) node [anchor=north west][inner sep=0.75pt]    {$z_{1}$};
% Text Node
\draw (274,228.4) node [anchor=north west][inner sep=0.75pt]    {$z_{2}$};
% Text Node
\draw (208,154.4) node [anchor=north west][inner sep=0.75pt]  [font=\footnotesize]  {$\theta $};
% Text Node
\draw (221,180.4) node [anchor=north west][inner sep=0.75pt]  [font=\footnotesize]  {$\phi $};
% Text Node
\draw (211,125.4) node [anchor=north west][inner sep=0.75pt]    {$r$};
% Text Node
\draw (221,208.4) node [anchor=north west][inner sep=0.75pt]    {$p$};
% Text Node
\draw (370,105.4) node [anchor=north west][inner sep=0.75pt]  [color={rgb, 255:red, 255; green, 0; blue, 0 }  ,opacity=1 ]  {$rp$};
% Text Node
\draw (525,84.4) node [anchor=north west][inner sep=0.75pt]  [color={rgb, 255:red, 255; green, 0; blue, 0 }  ,opacity=1 ]  {$z_{1} z_{2}$};
% Text Node
\draw (265,161.4) node [anchor=north west][inner sep=0.75pt]  [font=\footnotesize,color={rgb, 255:red, 255; green, 0; blue, 0 }  ,opacity=1 ]  {$\theta +\phi $};


\end{tikzpicture}

Since this is a change in position,
it doesn't matter from where we apply the change.
If we think of \( \vb{v} \) as our zero,
then \( \vb v + \va z = \vb v + ( \vb u - \vb v ) = \vb u \).

\section{Matrices}

A matrix is just a bunch of vectors put together.
Then matrix multiplication by a vector is just a linear combination of the matrix vectors
with the coefficients of the multiplying vector.
\[ Mv = 
    \begin{pmatrix}
    \vb{m_1}, \vb{m_2}, \dots, \vb{m_n}
    \end{pmatrix}
    v
    = v_1 \vb{m_1} + \dots + v_n \vb{m_n}.
\]
In other words,
we are treating the matrix vectors as a new kind of "basis" vectors or dimensions,
and we are describing a new vector with the same quantities along the new dimensions.
In this way, a matrix represents a new set of dimensions or matrix vectors,
each of which are still described with the standard basis.

\noindent
\noindent
\begin{tikzpicture}[x=0.75pt,y=0.75pt,yscale=-0.8,xscale=0.8]
%uncomment if require: \path (0,300); %set diagram left start at 0, and has height of 300

%Curve Lines [id:da16505242416400712] 
\draw [color={rgb, 255:red, 255; green, 0; blue, 0 }  ,draw opacity=1 ]   (103.15,205.12) .. controls (227.44,203.62) and (315.8,139.23) .. (321.79,37.4) ;
%Straight Lines [id:da7267676985584549] 
\draw    (101.65,247.05) -- (323.28,247.05) ;
%Straight Lines [id:da4645289606705647] 
\draw    (101.65,35.9) -- (101.65,247.05) ;
%Shape: Circle [id:dp39770385046406675] 
\draw   (189,192) .. controls (189,190.9) and (189.9,190) .. (191,190) .. controls (192.1,190) and (193,190.9) .. (193,192) .. controls (193,193.1) and (192.1,194) .. (191,194) .. controls (189.9,194) and (189,193.1) .. (189,192) -- cycle ;
%Straight Lines [id:da1118098783506678] 
\draw [color={rgb, 255:red, 0; green, 50; blue, 255 }  ,draw opacity=1 ]   (103.15,205.12) -- (256.5,160) ;
%Shape: Brace [id:dp33401170253306967] 
\draw   (102,208) .. controls (102,212.67) and (104.33,215) .. (109,215) -- (172.25,215) .. controls (178.92,215) and (182.25,217.33) .. (182.25,222) .. controls (182.25,217.33) and (185.58,215) .. (192.25,215)(189.25,215) -- (255.5,215) .. controls (260.17,215) and (262.5,212.67) .. (262.5,208) ;
%Curve Lines [id:da702343555310466] 
\draw [color={rgb, 255:red, 255; green, 0; blue, 0 }  ,draw opacity=1 ]   (408.15,203.12) .. controls (532.44,201.62) and (620.8,137.23) .. (626.79,35.4) ;
%Straight Lines [id:da5457712288003155] 
\draw    (406.65,245.05) -- (628.28,245.05) ;
%Straight Lines [id:da6145389316373155] 
\draw    (406.65,33.9) -- (406.65,245.05) ;
%Shape: Circle [id:dp8592795818420744] 
\draw   (494,190) .. controls (494,188.9) and (494.9,188) .. (496,188) .. controls (497.1,188) and (498,188.9) .. (498,190) .. controls (498,191.1) and (497.1,192) .. (496,192) .. controls (494.9,192) and (494,191.1) .. (494,190) -- cycle ;
%Straight Lines [id:da9754483079369636] 
\draw [color={rgb, 255:red, 0; green, 42; blue, 255 }  ,draw opacity=1 ]   (461.5,198) -- (619.5,76) ;
%Shape: Brace [id:dp4810786659509392] 
\draw   (463,205) .. controls (463,209.67) and (465.33,212) .. (470,212) -- (533.25,212) .. controls (539.92,212) and (543.25,214.33) .. (543.25,219) .. controls (543.25,214.33) and (546.58,212) .. (553.25,212)(550.25,212) -- (616.5,212) .. controls (621.17,212) and (623.5,209.67) .. (623.5,205) ;

% Text Node
\draw (97.9,262.4) node [anchor=north west][inner sep=0.75pt]    {$0$};
% Text Node
\draw (71,196.4) node [anchor=north west][inner sep=0.75pt]    {$1$};
% Text Node
\draw (193,193.4) node [anchor=north west][inner sep=0.75pt]    {$b$};
% Text Node
\draw (189,263.4) node [anchor=north west][inner sep=0.75pt]    {$1$};
% Text Node
\draw (178,155.4) node [anchor=north west][inner sep=0.75pt]  [color={rgb, 255:red, 0; green, 33; blue, 255 }  ,opacity=1 ]  {$b^{c}$};
% Text Node
\draw (177,222.4) node [anchor=north west][inner sep=0.75pt]    {$c$};
% Text Node
\draw (402.9,260.4) node [anchor=north west][inner sep=0.75pt]    {$0$};
% Text Node
\draw (376,194.4) node [anchor=north west][inner sep=0.75pt]    {$1$};
% Text Node
\draw (498,191.4) node [anchor=north west][inner sep=0.75pt]    {$b$};
% Text Node
\draw (494,261.4) node [anchor=north west][inner sep=0.75pt]    {$1$};
% Text Node
\draw (518,120.4) node [anchor=north west][inner sep=0.75pt]  [color={rgb, 255:red, 0; green, 24; blue, 255 }  ,opacity=1 ]  {$b^{c}$};
% Text Node
\draw (538,219.4) node [anchor=north west][inner sep=0.75pt]    {$c$};
% Text Node
\draw (258.5,163.4) node [anchor=north west][inner sep=0.75pt]    {$b^{c}$};
% Text Node
\draw (450,176.4) node [anchor=north west][inner sep=0.75pt]    {$a$};
% Text Node
\draw (621.5,79.4) node [anchor=north west][inner sep=0.75pt]    {$ab^{c}$};

\end{tikzpicture}

Multiplying two matrices together is just multiplying a matrix by multiple vectors
\[ MN = (\vb{m_1}, \dots, \vb{m_i})(\vb{n_1}, \dots, \vb{n_i})
    = ([n_{1_1} \vb{m_1} + \dots + n_{1_j} \vb{m_i}], \dots, 
       [n_{i_1} \vb{m_1} + \dots + n_{i_j} \vb{m_i}])
\]




\end{document}